\section{Introduction}


\noindent \red{[background]}




% previous methods & limitations
\noindent \red{[previous methods \& limitations]}

The conventional methods with a fixed iteration steps are infeasible to tackle the unknown scene without knowing the number of layers in advance.
It is also difficult for a human to count the number of layers in a scene exactly.
People take no efforts to know the relative position of two layers instead.
Therefore, a relative-order layer disentanglement in one stage is desirable.

\noindent \red{[our methods]}

We first express relative representation learning within the framework of generative modeling.
Then, we leverage the algorithm for inferring the relative representation.
Finally, we bring all elements together and propose an end-to-end model to obtain a disentangled representation for layer with relative positions.



\noindent \red{[our results]}




% contributions

The main contributions of this work can be summarized as follows:
\begin{itemize}
    \item 
\end{itemize}

\section{Related Work}

\noindent \textbf{Disentangled Representations Learning.}

\red{TODO}

\noindent \textbf{Unsupervised Scene Decomposition.}


MONet

IODINE

GENESIS

SPACE

GENESIS-v2



\red{TODO}

\section{Method}
In this section, we first express relative representation learning within the framework of generative modeling in Section~\ref{sec:relative_representation}.
Then, we leverage the algorithm for inferring the relative representation in Section~\ref{sec:inference} 
Finally, we bring all elements together and propose an end-to-end model to obtain a disentangled representation for layer with relative position in Section~\ref{sec:model}.

\subsection{Problem Setup}
In this part, we formally define the problem for layer disentanglement.

\red{TODO}

\para{Notations.} 




\subsection{Relative Representation}\label{sec:relative_representation}

\subsection{Inference}\label{sec:inference}

\subsection{\rev[XXNet]{A name}}\label{sec:model}

\section{Experiments}

\subsection{Setting}

\para{Dataset} \red{TODO}

\para{Architecture} \red{TODO}

\subsection{Metric1}
\red{[To prove the properties of proposed method]}
\subsection{Metric2}
\red{[To show the performances]}
\subsection{Metric3}
\red{[To show the limitations]}


\section{Ablation Study}

\red{TODO}

\section{Conclusion}

\red{TODO}